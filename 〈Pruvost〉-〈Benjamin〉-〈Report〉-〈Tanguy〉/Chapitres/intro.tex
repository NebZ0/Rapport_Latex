\chapter{Introduction}
\section{L'histoire du C++}

\begin{figure}[ht]
	\centering
	\includegraphics{Images/bjarne_stroustrup}
	\caption{Bjarne Stroustrup (2013)}
	\label{bjarne_stroustrup}
\end{figure}

\textbf{Le C++ a été créé au début des années 1980 par Bjarne Stroustrup (figure~\ref{bjarne_stroustrup})}, un ingénieur danois, alors qu'il travaillait chez Bell Labs. Il voulait améliorer le langage C en y ajoutant des fonctionnalités de programmation orientée objet, qui permettent de structurer le code en utilisant des concepts comme les classes et les objets. Le but était de combiner la puissance et l'efficacité du C avec une meilleure organisation pour écrire des programmes complexes. Aujourd'hui, C++ est largement utilisé pour développer des logiciels, des jeux vidéo, et même des systèmes embarqués.

\section{Différences entre C et C++}
\textbf{Le C et le C++ sont des langages de programmation étroitement liés}, mais ils diffèrent par leurs caractéristiques et leurs usages. Le C est un langage procédural qui se concentre sur les fonctions et les structures de données, idéal pour les systèmes bas-niveau comme les systèmes d'exploitation. En revanche, le C++ étend le C en introduisant la programmation orientée objet, avec des concepts comme les classes, l'héritage et le polymorphisme, ce qui le rend plus adapté à des projets complexes nécessitant une organisation modulaire. De plus, le C++ offre des fonctionnalités modernes comme les templates, les exceptions et la gestion automatique des ressources, absentes en C. Malgré leurs différences, \emph{le C++ reste compatible avec le C}, permettant d'utiliser du code C dans des projets C++.

\section{Installation de l'environnement de développement}
Afin d'écrire, compiler et exécuter du code C++ sur notre machine il est nécessaire d'installer deux outils essentiels:
\begin{itemize}
	\item Un éditeur de texte
	\item Un compilateur
\end{itemize}